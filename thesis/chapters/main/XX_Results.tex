\chapter{Results}\label{chap:Results}

% Update?
In this chapter, the results of the analysis are presented. \textbf{Chapter \ref{sec:Exploratory_Data_Analysis}} provides an overview of the data and explores the fairness of the dataset. 
\textbf{Chapter \ref{sec:Results}} presents the results of the analysis. \textbf{Chapter \ref{sec:Limitations}} discusses the limitations of the analysis.

\section{Exploratory Data Analysis}\label{sec:Exploratory_Data_Analysis}



\subsection{Data Overview}\label{subsec:Data_Overview}

The processed dataset contains two \textit{numerical} features: \textbf{interest\_rate} and \textbf{loan\_to\_value\_ratio}.
Their distributions can be seen in \textbf{Figure \ref{fig:CHXX_Numerical_Distributions_1}} and \textbf{Figure \ref{fig:CHXX_Numerical_Distributions_2}}.

\begin{figure}[h]
    \centering
    \caption{Boxplots of the Numerical Features}
    \includegraphics[width=0.85\textwidth]{CHXX_Numerical_Distributions_1.png}
    \caption*{The results of the KNNImputer applying mean (annotated in red) values for all missing values show clearly here by the narrow quartiles.}
    \label{fig:CHXX_Numerical_Distributions_1}
\end{figure}

\begin{figure}[h]
    \centering
    \caption{Histograms of the Numerical Features}
    \includegraphics[width=0.85\textwidth]{CHXX_Numerical_Distributions_2.png}
    \caption*{The results of the KNNImputer applying mean (annotated in red) values for all missing values show clearly here by the high amount of values at the mean.}
    \label{fig:CHXX_Numerical_Distributions_2}
\end{figure}

% Add analysis of categorical features

% Correlation Analysis?

\subsection{Fairness}\label{subsec:Fairness}

% Probably rename - tie to Hypothesis and Research Questions

Potential unfairness in the underlying data can be identified from assessing the distribution of the target variable across different groups.
\textbf{Figure \ref{fig:CHXX_Loan_Grant_By_Protected_Attribute}} shows the amount of (not) granted loans per race and by sex, the probabilities of being granted a loan across these groups can be found in \textbf{Table \ref{tab:loan_granting}}.\@

\begin{figure}[h]
    \centering
    \caption{Loan Grant by Protected Attribute}
    \includegraphics[width=0.85\textwidth]{CHXX_Loan_Grant_By_Protected_Attribute.png}
    \caption*{Discrimination by race is apparent in the data, as the amount of granted loans differs significantly}
    \label{fig:CHXX_Loan_Grant_By_Protected_Attribute}
\end{figure}

\begin{table}[htbp]
    \centering
      \caption{Loan Granting Statistics by Applicant Race and Sex}
      \begin{tabular}{lcc}
      \toprule
      \textbf{Applicant Race} & \textbf{Applicant Sex} & \textbf{Loan Granted (\%)} \\
      \midrule
      Black or African American & Male    & 46.4 \\
            & Female  & 44.1 \\
      White & Male    & 60.9 \\
            & Female  & 58.8 \\
      \bottomrule
      \end{tabular}
      \caption*{Regardless of their gender, Black or African American applicants are less likely to be granted a loan than White applicants.}
    \label{tab:loan_granting}%
\end{table}%

Even though the focus of the analysis is on the \textit{applicant\_race-1} attribute, \textit{applicant\_sex} has been included as a second discriminating factor, as it also constitutes a protected attribute.
Inspection of the results depicted here does however imply that the issue of racial equality is more pronounced than that of inequality between the sexes.
A chi-squared test of independence proves that assumption of underlying inequality between races in the data, as the p-value is \textit{<0.01} and therefore H0 (equality in granted loans) can be rejected at any significance level.
Utilizing the aforementioned \textbf{AIF360} package to assess the mean difference of granted loans between the races in the underlying data amounts to a \textit{14.9\%} difference.

\subsection{Enrichment Data}\label{subsec:Enrichment_Data}

% If required - probably describe here and put graphs in appendix

\section{Results}\label{sec:Results}        

As defined in \textbf{Chapter \ref{subsec:Performance_Assessment}} and \textbf{Chapter \ref{subsec:Fairness_Assessment}}, the model is evaluated based on XXX



\subsection{Initial Performance}\label{subsec:Initial_Performance}

Table XXX shows the results of the initial model run as described in \textbf{Chapter \ref{sec:Methodology}}.

XXX Hier Tabelle von finalem Durchlauf XXX

XXX Basic Kommentare XXX

As already mentioned in \textbf{Chapter \ref{subsec:Performance_Assessment}}, the model behaved somewhat different from the expectations: Imputation of missing values in the \textbf{debt\_to\_income\_ratio} feature led to a significant decrease in both fairness and performance measures of the model.
Therefore, it must be assumed that missingness in itself is not completely at random and holds information (or that the imputation method via a random forest regressor is not suitable for this task).

\subsection{Iteration I: Reweighing}\label{subsec:Iteration_I}

The results of the \textit{reweighed} model (see \textbf{Chapter \ref{subsec:Iterations}}) are promising. XXX

% Alle Ergebnisse einzeln oder am Ende Gesamttabelle mit allen Iterationen?

% \section{Limitations}\label{sec:Limitations} - hier oder am Ende?