\chapter{Discussion}\label{ch:discussion}

This thesis added a new perspective to the ongoing discussion of algorithmic fairness by combining the concepts of \textit{fairness} and \textit{explainability} in the context of mortgage lending, leveraging a recent dataset.

XXX

Although the explainability algorithms used did not find the potentially discriminatory protected attributes to be a major influence on the model’s decision-making (compare \textbf{figure \ref{fig:SHAP_beeswarm}, \ref{fig:LIME_Individual_Analyses}}, and \textbf{\ref{fig:Global_Surrogate}}), the patterns of the predicted data prove that the models do catch up on patterns within the data that promote inequality. 
The fact that all models, even those specifically corrected for fairness in the sensitive \textit{applicant\_race-1\_White} mimic the fact present in the underlying data that \textbf{White} applicants have proportionally higher chances of having a mortgage granted (see \textbf{table \ref{tab:loan_granting}}) proves that. 
It must therefore be assumed that just as So et al. \parencite{So2022} suggested in their study, there is a “deeper” inequality present in the data, resulting from \textbf{Black or African American} applicants historically being associated with worse attributes in terms of worthiness for a mortgage, even when the protected attribute itself is controlled for.
Even when ignoring \textit{applicant\_race-1\_White} itself as an influencing factor of the decision-making process, the fact that historic inequalities lead to \textbf{Black or African American} applicants living for example in poorer or less educated geographies reintroduces discrimination into the analysis.
Geography as a factor potentially impacting fairness has been analyzed in \textbf{chapter \ref{subsec:HMDA_EDA}}. While no causation can be assumed from these results alone, it is clear that regions with a higher proportion of \textbf{Black or African American} applicants are correlated with lower chance of having a mortgage granted, hinting towards a potential discrimination based on this feature being present in the data.

XXX

Revisiting the research questions of this thesis (see \textbf{chapter \ref{ch:Introduction}}), it must be concluded that the initial aim of this work could only be achieved in parts. 
While the exploratory data analysis, coupled with the analysis of the geographical enrichment data was useful to \textit{detect} and partly \textit{explain} unfairness within the data, the \textit{mitigation} of this unfairness proved hard to achieve. 
This might have been the case because of intrinsic unfairness being too deeply rooted within the data or the algorithms applied were not the optimal choices for this task.
Other possible causes of the dissatisfactory performance of the fairness adjustments might be the initial imbalance of the HMDA dataset with most of the applicants being white men (compare \textbf{chapter \ref{subsec:HMDA_EDA}}).

Nevertheless, XXX

While this thesis used a combination of different approaches to the topic of fairness, one way to truly improve the fairness of the models might be to include an even broader scope in terms of research focuses.
As Kim et al. \parencite{Kim2023} suggest in their review of fairness in credit, true multidisciplinary research is needed to tackle the issue of fairness in credit scoring.