\section{Data}\label{sec:Data}

\subsection{HMDA Data}\label{subsec:HMDA_Data}

The main dataset is retrieved from the publicly available HMDA Data Browser\footnote{The Data Browser can be found \href{https://ffiec.cfpb.gov/data-browser/data/2022?category=states}{via this link}.}, a database on mortgage applications, originally created in 1975 to tackle discrimination against low-income borrowers \parencite{Bogen2020}. 
The data selection process is loosely based on the paper by Ghoba and Colaner \parencite{Ghoba}, who use counterfactuals to tackle the issue of fairness in a 2019 version of the HDMA Dataset. 
All data were generated in \textbf{2022}, which supports the claim of this thesis to provide data that are newer than the studies available and potentially already include effects from the COVID pandemic. \textbf{All financial institutions} were included, but to narrow down the research focus to racial fairness specifically, the approach of Ghoba and Colaner was followed by only including \textbf{non-Hispanic White} and \textbf{Black or African Americans} in the analysis. 
In terms of geographical filtering, inspiration was drawn from the paper by Singh et al. \parencite{Singh2022}, but instead of including different-sized states, a fairness-based approach was taken: 
U.S.News publishes a ranking of US states by their equality, using a somewhat scientific approach to measure distributions by race, gender and other fairness-related aspects and aggregate them to overall equality levels\footnote{The 2023 ranking can be found \href{https://www.usnews.com/news/best-states/rankings/opportunity/equality?sort=rank-desc}{here}, the corresponding methodology \href{https://www.usnews.com/news/best-states/articles/methodology}{here}}. 
To use a subset of the HMDA data that is likely to include fairness issues, the five states considered least equal in the year 2023 are included in the data, being \textbf{South Dakota, Louisiana, Utah, Texas,} and \textbf{Montana}. The resulting dataframe has \textit{867.401 rows and 99 columns}.

The features included for the analysis are close to the ones proposed by Ghoba and Colaner to maintain a degree of comparability, but adjusted to the scope of this thesis by including a different target variable as well as different geographical features: 
\textit{action\_taken, county\_code, interest\_rate, applicant\_sex, applicant\_race-1, loan\_type, debt\_to\_income\_ratio, loan\_to\_value\_ratio,} and \textit{lien\_status}\footnote{Datasheets can be found under \href{https://ffiec.cfpb.gov/documentation/publications/loan-level-datasets/lar-data-fields}{this link}}. 
Missing values were present in \textit{interest\_rate (320.335), loan\_to\_value\_ratio (257.408), debt\_to\_income\_ratio (229.442),} and \textit{county\_code (9.053)}. 
Furthermore, three columns contained the string \textit{Exempt} as a value \textit{(interest\_rate: 20.591 occurrences; loan\_to\_value\_ratio, and debt\_to\_income\_ratio: 20.533 occurrences each)}, preventing them from initially being cast as numerical types.
In order to clean and reshape the data in a easily analyzable format, several steps were taken that are described in the following and summarized in \textbf{table \ref{tab:HMDA_transformation_summary}}.

\begin{table}[h]
    \centering
    \begin{tabularx}{\textwidth}{l *{3}{>{\centering\arraybackslash}X}}
    \hline
     & \textbf{Transformation Step} & \textbf{Reasoning} \\
    \hline
    \textbf{Rows with missing county code} & Dropping & insignificant amount, imputation not feasible, presumably MCAR \\
    \textbf{All variables} & Typecasting & Required for further analysis \\
    \textbf{loan\_granted} & Creation & Target variable for the classification task \\
    \textbf{Exempt} & Recoding & Recoding to zero for proper typecasting \\
    \textbf{debt\_to\_income\_ratio} & Binning & Reduction of total categories \\
    \textbf{loan\_to\_value\_ratio} & Outlier Removal & Few outliers skewed the initial distribution \\
    \textbf{interest\_rate} and \textbf{loan\_to\_value\_ratio} & Imputation & Mean Imputation to tackle missingness \\
    \hline
    \end{tabularx}
    \caption{Transformation Steps in the HMDA Mortgage Data}
    \small
    Several transformation steps have been applied to the HMDA mortgage data in order to make it easily analyzable.
    \label{tab:HMDA_transformation_summary}
\end{table}

All rows including missing values for the \textit{county\_code} were \textbf{dropped}, as their relative amount was insignificant, imputation was not logically possible for this variable, and missingness completely at random could be assumed from visual inspection using the missingno package (see \textbf{Figure~\ref{fig:CH03_Missingno_Completeness}}). 
All features were \textbf{cast} to their appropriate types \textit{(county\_code: string; applicant\_race-1, applicant\_sex, lien\_status, loan\_type: category)}. Furthermore, a \textbf{new target variable} \textit{(loan\_granted)} was created from \textit{action\_taken} (which was dropped), assuming 1 for granted loans and 0 for denials instead of including different reasons for (dis-)approval.

\begin{figure}[h]
    \centering
    \includegraphics[width=0.85\textwidth]{images/CH03_Missingno_Completeness.png}
    \caption{Visual Inspection of Missingness in the missingno Package}
    \medskip
    \small
    Visual Inspection of Missingness shows that some datapoints seem to be missing at random in \textit{interest\_rate}, \textit{loan\_to\_value\_ratio}, and \textit{debt\_to\_income\_ratio}, while the other features expose no or very little missingness.
    \label{fig:CH03_Missingno_Completeness}
\end{figure}

All \textit{Exempt} strings were \textbf{recoded} as zero, allowing proper typecasting (float) for the \textit{interest\_rate} and \textit{loan\_to\_value\_ratio} variables. 
For \textit{\_to\_income\_ratio}, further \textbf{binning} was introduced, creating the new categories \textit{36\%-41\%, 41\%-45\%}, and \textit{46\%-49\%}. %Initially, Missingness was encoded as a separate category, bringing the number of distinct values for \textit{debt\_to\_income\_ratio} to eleven. 
%This was kept, because later analysis showed that missingness in the \textit{debt\_to\_income\_ratio} was a feature that heavily impacted predictions: As a comparison, a random forest classifier was trained to predict the missingness in the \textit{debt\_to\_income\_ratio} based on the other features.
%Although only a 63.7\% accuracy was achieved on the test data, the model was used to impute the missing values in the \textit{debt\_to\_income\_ratio} column, as this methodology still is more thorough than other imputation methods for categorical values like using the mode. As per the predictions, most of the missing values (125,312) were imputed as \textit{>61\%}, followed by \textit{50\%-60\%}. 
%Because later analysis proved that encoding as missingness was way more beneficial to the performance metrics, this option was finally included.
As later analysis proved that missingness in the \textit{debt\_to\_income\_ratio} was a feature that heavily impacted predictions, the missingness was encoded as a separate category.
After assessing the distributions of \textit{interest\_rate} and \textit{loan\_to\_value\_ratio}, extreme outlier values \textit{(loan\_to\_value\_ratio > 250)} were \textbf{dropped}. 

To tackle missingness in \textit{interest\_rate} and \textit{loan\_to\_value\_ratio}, \textbf{imputation} was utilized. 
As the preferred way of imputing, the KNNImputer from the sklearn package, proved to be too computationally expensive to be efficiently used, the IterativeImputer from the same package was used as an alternative. 
This multivariate imputing technique is supposed to infer feature values from other features\footnote{Further documentation can be found via \href{https://scikit-learn.org/stable/modules/generated/sklearn.impute.IterativeImputer.html}{this link}}. It is however set up to default to the mean as the imputation value if no satisfying solution is found, which was the case in the data.

\subsection{Enrichment Data}\label{subsec:Enrichment_Data}

The enrichment data was obtained from the \textit{\href{https://www.ers.usda.gov/data-products/county-level-data-sets/}{USDA ERS page}}\footnote{See \textit{\href{https://www.ers.usda.gov/data-products/county-level-data-sets/}{this link}}}. All data used are structured, tabular data that are publicly available. Privacy concerns are not relevant, as the data is anonymized and aggregated. All available reports were downloaded, specifically the following datasets:

\begin{itemize}
    \item \textbf{Poverty} (2021 latest)
    \item \textbf{Population} (2022 latest)
    \item \textbf{Unemployment, and Median Household Income} (annual average 2022 unemployment and 2021 median income latest)
    \item \textbf{Education} (2017–21, 5-year average latest).
\end{itemize}

In order to clean and reshape the data in a easily analyzable format, several steps were taken that are described in the following and summarized in \textbf{table \ref{tab:enrichment_transformation_summary}}. %\footnote{For all detail, see the corresponding cleaning notebook at !!! XXX !!!}:

\begin{table}[h]
    \centering
    \begin{tabularx}{\textwidth}{l *{3}{>{\centering\arraybackslash}X}}
    \hline
     & \textbf{Transformation Step} & \textbf{Reasoning} \\
    \hline
    \textbf{Geography} & Filtering & Only include the five states in question \textit{("SD", "LA", "UT", "TX", "MT")} \\
    \textbf{Year} (where applicable) & Filtering & Only include the newest datapoints \\
    \textbf{Features} & Reducing & Only include the most relevant features \textit{(Poverty, Population, Unemployment and Median Household Income, Education)} \\
    \textbf{Data Structure} & Pivoting & Use the attributes as feature names \\
    \textbf{Indexing} & Indexing & Use the FIPS code and the name of the respective county as index \\
    \textbf{Renaming} & Renaming & Rename the feature columns for clarity \\
    \textbf{Merging} & Merging & Merge all datasets into a single dataframe \\
    \hline
    \end{tabularx}
    \caption{Transformation Steps in the Enrichment Data}
    \small
    Several transformation steps have been applied to the geographical enrichment data in order to make it easily analyzable.
    \label{tab:enrichment_transformation_summary}
\end{table}

All datasets were \textit{filtered} to only contain county-level data for the five states in question \textit{("SD", "LA", "UT", "TX", "MT")}, not the aggregated values for the full states, and, in case multiple years of analysis were available, to only include the newest datapoints. 
Where there were more features available than would be useful for the analysis, the datasets were \textit{reduced} to only include the most relevant features, being:

\begin{itemize}
    \item \textbf{Poverty:} Only the percentage of the population living in poverty (PCTPOVALL\_2021) was included
    \item \textbf{Population:} Only the total population (POP\_ESTIMATE\_2022) was included
    \item \textbf{Unemployment, and Median Household Income:} The unemployment rate (Unemployment\_rate\_2022) and the median household income (Median\_Household\_Income\_2021) were included
    \item \textbf{Education:} All relative values, i.e.\ percentages of adults with their corresponding highest degrees were included.
\end{itemize}

All datasets were \textit{pivoted} in order to use the attributes as feature names and \textit{indexed} by the FIPS code and the name of the respective county. 
After basic \textit{checks for completeness}, the feature columns were \textit{renamed} for clarity. Finally, all datasets were \textit{merged} into a single dataframe, which was then \textit{exported} as a pickle file for further use in the analysis.

The cleaned data are \textit{complete} in a way that there are values available for all features and for all of the 469 counties. Basic summary statistics can be found in \textbf{table \ref{tab:enrichment_summary}}. 
Across all 469 counties that are included in this analysis, a high school degree is the most common highest degree, with the mean of population percentages with that degree being \textbf{32.9\%} overall, followed by a college degree with \textbf{30.9\%}. 
A high discrepancy in the educational standard between the different observational units becomes apparent when considering the presence of counties where \textbf{81.6\%} of the population have less than a high school degree, while in others, \textbf{55.2\%} have a bachelors degree or higher.
In terms of population, the counties in this analysis are heterogenous as well, with the mean population being \textbf{85.36 thousand} with standard deviation of \textbf{318.70 thousand} and the most inhabited county having nearly 100,000 times as many inhabitants as the least inhabited.
The socio-economic factors expose a wide range of values as well, with the poverty rate ranging from \textbf{3.9\%} to \textbf{43.5\%}, the median household income from \textbf{25.65 thousand} to \textbf{124.35 thousand}, and the unemployment rate from \textbf{0.6\%} to \textbf{11.0\%}.

\begin{table}[h]
    \centering
    \begin{tabularx}{\textwidth}{l *{5}{>{\centering\arraybackslash}X}}
    \hline
     & \textbf{Count} & \textbf{Mean} & \textbf{Std} & \textbf{Min} & \textbf{Max} \\
    \hline
    College Degree & 469.00 & 30.9\% & 6.03 & 0.00\% & 76.92\% \\
    Bachelor Degree or Higher & 469.00 & 21.7\% & 8.42 & 0.00\% & 55.17\% \\
    High School Degree & 469.00 & 32.9\% & 6.60 & 12.93\% & 51.17\% \\
    Less than High School Degree & 469.00 & 14.5\% & 8.18 & 0.60\% & 81.55\% \\
    Population (thousands) & 469.00 & 85.36 & 318.70 & 0.05 & 4,780.91 \\
    Poverty Rate & 469.00 & 15.9\% & 6.23 & 3.90\% & 43.50\% \\
    Median Household Income (thousands) & 469.00 & 56.6 & 13.94 & 25.65 & 124.35 \\
    Unemployment Rate & 469.00 & 3.6\% & 1.21 & 0.60\% & 11.00\% \\
    \hline
    \end{tabularx}
    \caption{Summary Statistics of the Enrichment Data}
    \small
    The summary statistics show that the data is complete and has a wide range of values.
    \label{tab:enrichment_summary}
\end{table}

%A closer look into the discrepancies between the highest and lowest values of selected enrichment features can be found in \textbf{figure \ref{fig:CH03_Geo_High_Low}}.
%In terms of population, the high discrepancy betwen lowest and highest value is pronounced by the fact that the highest populated \textit{Harris County} has nearly twice as many inhabitants as the second-largest \textit{Dallas County}, while the lowest populated \textit{Loving County} has less than a fourthof the inhabitants of the second-smallest \textit{King County}.
%Similar outlier values can be found in the unemployment rate with \textit{Loving county} (0.6\%) differing strongly from 

%\begin{figure}[h]
%    \centering
%    \includegraphics[width=0.85\textwidth]{images/CH03_Geographic_Low_vs_High.png}
%    \caption{Highest and Lowest Values of Selected Enrichment Features}
%    \medskip
%    \small
%    There is a wide range in the numerical enrichment features, creating some natural outliers.
%    \label{fig:CH03_Geo_High_Low}
%\end{figure}

Expectedly, there is a high correlation between some of the features (see \textbf{figure \ref{fig:CH03_Enrichment_Correlation}}).
While a bachelors degree or higher tends to be associated with a higher median household income (positive correlation of \textbf{0.62}), a high school degree or less than a high school degree as a highest degree is associated with a higher poverty rate (positive correlation of \textbf{0.33} and \textbf{0.54} respectively).
A similar relation of socio-economic factors with education can be observed in the correlation of the unemployment rate with the highest degree, with a negative correlation of \textbf{-0.42} for holders of a bachelors degree or higher, compared to a positive correlation of \textbf{0.53} for less than a high school degree, implying that on average, observational units with less than a high school degree a more likely to be unemployed than those with a bachelors degree or higher.

\begin{figure}[h]
    \centering
    \includegraphics[width=0.85\textwidth]{images/CH03_Enrichment_Correlation.png}
    \caption{Correlation Between the Enrichment Features}
    \medskip
    \small
    Some of the enrichment features expectedly show a high correlation, an example being \textit{median\_household\_income} and \textit{poverty\_rate}.
    \label{fig:CH03_Enrichment_Correlation}
\end{figure}