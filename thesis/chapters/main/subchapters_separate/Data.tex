\section{Data}\label{sec:Data}

Test

\subsection{HMDA Data}\label{subsec:HMDA_Data}

The main dataset is retrieved from the publicly available HMDA Data Browser\footnote{The Data Browser can be found \href{https://ffiec.cfpb.gov/data-browser/data/2022?category=states}{via this link}.}, a database on mortgage applications, originally created in 1975 to tackle discrimination against low-income borrowers \parencite{Bogen2020}. 
The data selection process is loosely based on the paper by Ghoba and Colaner \parencite{Ghoba}, who use counterfactuals to tackle the issue of fairness in a 2019 version of the HDMA Dataset. 
All data were generated in \textbf{2022}, which supports the claim of this thesis to provide data that are newer than the studies available and potentially already include effects from the COVID pandemic. \textbf{All financial institutions} were included, but to narrow down the research focus to racial fairness specifically, the approach of Ghoba and Colaner was followed by only including \textbf{non-Hispanic White} and \textbf{Black or African Americans} in the analysis. 
In terms of geographical filtering, inspiration was drawn from the paper by Singh et al. \parencite{Singh2022}, but instead of including different-sized states, a fairness-based approach was taken: 
U.S.News publishes a ranking of US states by their equality, using a somewhat scientific approach to measure distributions by race, gender and other fairness-related aspects and aggregate them to overall equality levels\footnote{The 2023 ranking can be found \href{https://www.usnews.com/news/best-states/rankings/opportunity/equality?sort=rank-desc}{here}, the corresponding methodology \href{https://www.usnews.com/news/best-states/articles/methodology}{here}}. 
To use a subset of the HMDA data that is likely to include fairness issues, the five states considered least equal in the year 2023 are included in the data, being \textbf{South Dakota, Louisiana, Utah, Texas,} and \textbf{Montana}. The resulting dataframe has \textit{867.401 rows and 99 columns}.

The features included for the analysis are close to the ones proposed by Ghoba and Colaner to maintain a degree of comparability, but adjusted to the scope of this thesis by including a different target variable as well as different geographical features: 
\textit{action\_taken, county\_code, interest\_rate, applicant\_sex, applicant\_race-1, loan\_type, debt\_to\_income\_ratio, loan\_to\_value\_ratio,} and \textit{lien\_status}\footnote{Datasheets can be found under \href{https://ffiec.cfpb.gov/documentation/publications/loan-level-datasets/lar-data-fields}{this link}}. 
Missing values were present in \textit{interest\_rate (320.335), loan\_to\_value\_ratio (257.408), debt\_to\_income\_ratio (229.442),} and \textit{county\_code (9.053)}. 
Furthermore, three columns contained the string \textit{Exempt} as a value \textit{(interest\_rate: 20.591 occurrences; loan\_to\_value\_ratio, and debt\_to\_income\_ratio: 20.533 occurrences each)}, preventing them from initially being cast as numerical types.

All rows including missing values for the \textit{county\_code} were \textbf{dropped}, as their relative amount was insignificant, imputation was not logically possible for this variable, and missingness completely at random could be assumed from visual inspection using the missingno package (see \textbf{Figure~\ref{fig:CH03_Missingno_Completeness}}). 
All features were \textbf{cast} to their appropriate types \textit{(county\_code: string; applicant\_race-1, applicant\_sex, lien\_status, loan\_type: category)}. Furthermore, a \textbf{new target variable} \textit{(loan\_granted)} was initiated from \textit{action\_taken} (which was dropped), assuming 1 for granted loans and 0 for denials instead of including different reasons for (dis-)approval.

\begin{figure}[h]
    \centering
    \includegraphics[width=0.85\textwidth]{images/CH03_Missingno_Completeness.png}
    \caption{Visual Inspection of Missingness in the missingno package}
    \label{fig:CH03_Missingno_Completeness}
\end{figure}

All \textit{Exempt} strings were \textbf{recoded} as zero, allowing proper typecasting (float) for the \textit{interest\_rate} and \textit{loan\_to\_value\_ratio} variables. 
For \textit{debt\_to\_income\_ratio}, further \textbf{binning} was introduced, creating the new categories \textit{37\%-41\%, 41\%-45\%}, and \textit{46\%-49\%}. Missingness was encoded as a separate category, bringing the number of distinct values for \textit{debt\_to\_income\_ratio} to eleven. 
After assessing the distributions of \textit{interest\_rate} and \textit{loan\_to\_value\_ratio}, extreme outlier values \textit{(loan\_to\_value\_ratio > 250)} were \textbf{dropped}. 

To tackle missingness in \textit{interest\_rate} and \textit{loan\_to\_value\_ratio}, \textbf{imputation} was utilized. 
As the preferred way of imputing, the KNNImputer from the sklearn package proved to be too computationally expensive to be efficiently used, the IterativeImputer from the same package was used as an alternative. 
This multivariate imputing technique is supposed to infer feature values from other features\footnote{Further documentation can be found via \href{https://scikit-learn.org/stable/modules/generated/sklearn.impute.IterativeImputer.html}{this link}}. It is however set up to default to the mean as the imputation value if no satisfying solution is found, which was the case in the data.

\subsection{Enrichment Data}\label{subsec:Enrichment_Data}

Test