\chapter{Data and Methodology}\label{ch:Data_and_Methodology}



\section{Data}\label{sec:Data}



\subsection{HMDA Data}\label{subsec:HMDA_Data}

% Follow structure from presentation Ana: From where, why, which steps etc. - probably add in appendix the code for the enrichment?

% Geography: California - Why? (Large state, high amount of data - quote results from summary table)
% Time: 2022 (!)
% No filter on financial institutions - Why?
% Filter: "Single Family (1-4 Units):Site-Built" - Why?

The underlying data for this analysis was obtained \textit{\href{https://ffiec.cfpb.gov/data-browser/data/2022?category=states}{via the HMDA Data Browser}}.

\subsection{Enrichment Data}\label{subsec:Enrichment_Data}

% Follow structure from presentation Ana: From where, why, which steps etc. - probably add in appendix the code for the enrichment?

% See ydata report

The enrichment data was obtained from the \textit{\href{https://www.ers.usda.gov/data-products/county-level-data-sets/}{USDA ERS page}}. All available reports were downloaded, specifically the following datasets:

\begin{itemize}
    \item \textbf{Poverty} (2021 latest)
    \item \textbf{Population} (2022 latest)
    \item \textbf{Unemployment, and Median Household Income} (annual average 2022 unemployment and 2021 median income latest)
    \item \textbf{Education} (2017–21, 5-year average latest).
\end{itemize}

In order to clean and reshape the data in a easily analyzable format, the following steps were taken\footnote{For all detail, see the corresponding cleaning notebook at !!! XXX !!!}:
All datasets were \textit{filtered} to only contain county-level data, not the aggregated values for the full state and, in case multiple years of analysis were available, to only include the newest datapoints. 
Where there were more features available than would be useful for the analysis, the datasets were \textit{reduced} to only include the most relevant features, being:

\begin{itemize}
    \item \textbf{Poverty:} Only the percentage of the population living in poverty (PCTPOVALL\_2021) was included
    \item \textbf{Population:} Only the total population (POP\_ESTIMATE\_2022) was included
    \item \textbf{Unemployment, and Median Household Income:} The unemployment rate (Unemployment\_rate\_2022) and the median household income (Median\_Household\_Income\_2021) were included
    \item \textbf{Education:} All relative values, i.e.\ percentages of adults with their corresponding highest degrees were included.
\end{itemize}

All datasets were \textit{pivoted} in order to use the attributes as feature names and \textit{indexed} by the FIPS code and the name of the respective county. After basic \textit{checks for completeness}, the feature columns were \textit{renamed} for clarity. Finally, all datasets were \textit{merged} into a single dataframe, which was then \textit{exported} as a pickle file for further use in the analysis.

\section{Methodology}\label{sec:Methodology}