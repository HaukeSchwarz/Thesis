\chapter*{Abstract Portuguese}\label{abstractp}

\noindent
\textbf{Título:} Avaliando a Justiça das Decisões de Empréstimos Hipotecários: \\
Utilizando Explicabilidade para Quantificar o Compromisso entre Justiça e Desempenho em Tomadas de Decisão Algorítmicas\\
\textbf{Autor}: Hauke Schwarz
\vspace{1em}

O aumento na aplicação de tomada de decisão algorítmica tem gerado preocupações sobre a justiça das decisões automatizadas. Em áreas com impacto significativo sobre os indivíduos, como saúde, reincidência e finanças, é crucial entender minuciosamente os processos de tomada de decisão, tanto do ponto de vista regulatório quanto moral.

No entanto, melhorar o desempenho preditivo dos algoritmos muitas vezes aumenta a sua complexidade e reduz a sua compreensibilidade. Isto é particularmente verdadeiro para modelos de “caixa-negra", como as redes neurais, que carecem de interpretabilidade direta.

Esta tese examina se os dados de concessão de hipotecas de 2022, que têm como base o Home Mortgage Disclosure Act, podem ser usados para treinar uma rede neural para prever a aprovação de hipotecas com base em dados demográficos, com foco em geografia e atributos sensíveis, como a raça do solicitante.

A adaptação iterativa do modelo de redes neurais inicial, utilizando diferentes algoritmos focados na justiça e a aplicação de várias técnicas de explicabilidade, produziu resultados mistos. Embora alguns resultados tenham sido promissores no âmbito desta tese, nenhum melhorou significativamente a justiça ou o desempenho preditivo do modelo proposto.

Tal deve-se provavelmente a fatores discriminatórios subjacentes nos dados que não podem ser mitigados apenas controlando a raça dos solicitantes, destacando uma área importante para pesquisas futuras.

\vspace{3em}

\textbf{Palavras-Chave:} Interpretabilidade, Explicabilidade, Justiça na tomada de decisões algorítmicas \\